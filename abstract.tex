\noindent
Color Transparency (CT) is a prediction of QCD that at high momentum transfer
$Q ^2$, a system of quarks which would normally interact strongly with nuclear
matter could form a small color-neutral object whose compact transverse size
would be maintained for some distance, passing through the nucleus undisturbed.
A clear signature of CT would be a dramatic rise in nuclear transparency $T$
with increasing $Q^2$.
CT emerges as a deviation from Glauber multiple scattering
theory, which predicts constant $T$.
While a rise in nuclear transparency would provide an unequivocal validation of
QCD factorization theorems, the complex nature of nuclear
interactions renders its observation difficult to predict.
The E12-06-107 experiment at JLab measured $T$ in quasielastic electron-proton
scattering with carbon-12 and liquid hydrogen targets, for $Q^2$ between 8.0
and \SI{14.2}{\giga\electronvolt\squared}, a range over which models of CT
predicted that $T$ might differ appreciably from Glauber calculations.
Supported in part by US DOE grant DE-FG02-03ER41240.
