\section{The Onset of Color Transparency}
Predicts a rise in $T$ above some threshold $Q^2_0$.

\subsection{Frankfurt}
Frankfurt~\cite{Frankfurt_1995_PRC}.

\subsection{Cosyn}
Cosyn~\cite{Cosyn_2006}. RMSGA + CT (quantum diffusion)

Similar to eikonal approximation of equation~\ref{eqn:eikonal_approximation}.
After scattering, the outgoing wavefunction $\psi_{out}$ picks up a complex
phase $\chi(\vec{r})$.
\begin{equation}
    \psi_{out}(\vec{r}) = e^{i\chi(\vec{r})} \psi_{in}(\vec{r})
\end{equation}
Cosyn et al. introduce a Gaussian \textit{grey disc} profile function
$\Gamma(\vec{b})$ to parameterize this phase as a function of impact parameter
$\vec{b}$.
For a single rescattering,
\begin{equation}
    \psi_{out}(\vec{r}) = (1-\Gamma(\vec{b})) \psi_{in}(\vec{r})
\end{equation}
The profile function for a hadron $h$ scattering from a nucleon $N$ is
determined by
the interaction cross section $\sigma_{hN}^{tot}$,
the ratio of the real and imagniary parts of the scattering amplitude $\epsilon_{hN}$,
and
slope parameter $\beta_{hN}$,
all of which are momentum-dependent.
\begin{equation}
    \Gamma_{hN}(\vec{b}) =
        \frac{\sigma_{hN}^{tot}(1-i\epsilon_{hN})}
             {4\pi\beta_{hN}^2}
        \exp{-\frac{\vec{b}^2}{2\beta_{hN}^2}}
\end{equation}
Using the same frozen approximation as the Glauber approximation, the total
phase shift for $N_{scatter}$ rescatterings is simply a product
\begin{equation}
    e^{i\chi(\vec{r})} = \prod_i^{N_{scatter}} \left(1-\Gamma_i(\vec{b_i})\right)
\end{equation}


To include CT in the model, Cosyn et al. replace
the total cross section $\sigma_{hN}^{tot}$
with
an effective cross section $\sigma_{hN}^{eff}$.
based on the quantum diffusion model~\cite{Farrar_1988},
which accounts for reduced interaction between the prehadron and nuclear matter
over a hadron formation length $l_h$,
\begin{equation}
    \sigma_{hN}^{eff} = \sigma_{hN}^{tot}
    \left\{
        \left[\frac{z}{l_h} +
               \frac{\left\langle n^{2} k_{t}^{2}\right\rangle}{t} \left(1-\left(\frac{z}{l_h}\right)\right)
        \right]
        \theta\left(l_h-z\right) +
        \theta\left(z-l_h\right)
    \right\}
\end{equation}
In this expression,
$n$ is the number of valence quarks (2 for mesons, 3 for baryons),
$k_t=\SI{0.35}{\giga\electronvolt}$ is the average transverse momentum of a quark inside a hadron,
$z$ is the distance the obejct has traveled since its creation,
and
$l_h=2p/\Delta M^2$ is the hadronic expansion length.
This length depends on
the momentum $p$ of the outgoing hadron
and
the mass squared difference between the prehadron and outgoing hadron state.
Cosyn et al. use $\Delta M^2 = \SI{1}{\giga\electronvolt\squared}$ for protons.
