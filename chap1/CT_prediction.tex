\section{The Onset of Color Transparency}
Predicts a rise in $T$ above some threshold $Q^2_0$.
Both Frankfurt et al. and Cosyn et al. use the quantum diffusion model

\subsection{Frankfurt}
Frankfurt~\cite{Frankfurt_1995_PRC}.

Frankfurt et al. derive an modified nuclear density $\tilde{\rho}(r)$ that
takes nucleon correlations into account based on single nucleon density
functions $\rho(r)$ and two-body correlation functions $g_h(r,r')$,
\begin{equation}
    \tilde{\rho}(r)
        = (A-1) \rho(r)
          \left[
               1 + C_h(r_1,r) - \frac{A-1}{2} \int_{z_1} \Gamma(b_1-b')g_h(r,r')\rho(r')d^3r'
          \right]
\end{equation}
Here, $C_h(i,j)$ are two-nucleon correlation factors that are functions of the
distance $r_{ij}=|\vec{r}_j-\vec{r}_i|$ between nucleons $i$ and
$j$~\cite{Weise_1972}.
The single nucleon wave functions $\phi_h(r)$ are the overlap integral between
wave functions of the $A$-body ground state wave function and the $(A-1)$-body
recoil nucleus.
Their normalization requires $\int|\phi_h(r)|d^3r=1$ and
$\rho(r) = \sum_h \omega_h^2 \left| \phi_h(r) \right|$ where $\omega_h$ is the
occupation probability of the orbital $h$.

\begin{equation}
    \Gamma_i(\vec{b}) = \frac{1}{2\pi i k}
                  \int e^{i\vec{k}_t \cdot \vec{b}} f(\vec{k}_t) d^2k_t
\end{equation}

\begin{equation}
    f(k_t) = \left(\frac{k_t}{4\pi}\right)^2
             \sigma_{tot}^2
             (1+\alpha^2) e^{bt}
\end{equation}

\begin{equation}
    f_{CT}(k_t, z, Q^2) = i\frac{k}{4\pi} \sigma_{eff}(z,Q^2) e^{bt/2}
                          \frac{G_N\left( t \sigma_{eff}(z,Q^2)/\sigma_{eff} \right)}
                               {G_N\left( t \right)}
\end{equation}

\begin{equation}
    \mathcal{M}_{h}^{\gamma^*A} = \int d^3 r_1 \omega_h \phi_h\left(r_{1}\right)
                                  \hat{O}^{\mathrm{em}}\left(Q^{2}\right)
                                  e^{-i \vec{p}_p \cdot \vec{r}_1}
                                  \exp{- \int_{z_{1}}
                                  \Gamma\left(b_{1}-b\right) \tilde{\rho}(r) d^3 r}
\end{equation}

In the distorted wave impulse approximation (DWIA) the cross section can be
written,
\begin{equation}
    \frac{d^6\sigma}{dE'_{e} d\Omega'_{e} d^3p'_{p}} = p'_p E'_p \sigma_{eN} S(\vec{p}_p, E_M, \vec{p'}_p)
\end{equation}
where $\sigma_{eN}$ is the cross section for an electron scattering from a
bound nucleon and $S(\vec{p}_p, E_M, \vec{p'}_p)$ is the distorted
spectral function.
For a fixed shell $h$ the spectral function can be written~\cite{Frullani_1984},
\begin{equation}
    S(\vec{p}_p, E_M, \vec{p'}_p) = n_h(E_m) |\Phi_h(\vec{p}_p, \vec{p'}_p)|^2
\end{equation}
where $n_h(E_m)\propto\omega_h^2$ characterizes the strength of the shell and
$\Phi_h(\vec{p}_p, \vec{p'}_p)$ is the distorted momentum distribution for
nucleons in the $h$ shell.
Using the above matrix amplitude,
\begin{equation}
    \left| \Phi_h(\vec{p}_p, \vec{p'}_p)\right|^2
        = \left|
            \int d^3 r_1 \Psi_h(r_1) e^{-i \vec{p}_p \cdot \vec{r}_1}
            \exp{- \int_{z_{1}} \Gamma\left(b_{1}-b\right) \tilde{\rho}(r) d^3 r}
          \right|
\end{equation}

The nuclear transparency for the $h$ shell is
\begin{equation}
    T_h = \left(\frac{\sigma_{exp}}{\sigma_{PWIA}}\right)
        = \frac{|\Phi^{DWIA}_h(p_p,p'_p)|^2}
               {|\Phi^{PWIA}_h(p_p)|^2}
\end{equation}

The ground state wavefunctions Frankfurt et al. use are calculated in the
Skyrme-Hartree-Fock model with correlated interactions~\cite{Reinhard_1991}.

Uses the same effective cross section as Cosyn. So I should probably factor
that out of the next subsection and put it before this one as common material.

\begin{figure}[h]
    \centering
    \begin{subfigure}[b]{0.45\textwidth}
        \centering
        \includegraphics[width=\textwidth]{chap1/frankfurt_transparency_without_CT.pdf}
        % \caption{X plane}
        \label{fig:frankfurt_transparency_without_CT}
    \end{subfigure}
    % \hfill
    \begin{subfigure}[b]{0.45\textwidth}
        \centering
        \includegraphics[width=\textwidth]{chap1/frankfurt_transparency_with_CT.pdf}
        % \caption{U plane}
        \label{fig:frankfurt_transparency_with_CT}
    \end{subfigure}
    \caption{Transparency calculations for ${}^{12}C(e,e'p)$ based on a model
             that accounts for nucleon correlations and proton knock-out from
             particular nuclear shells~\cite{Frankfurt_1995_PRC}.
             The figure on the left is for a model that does not include CT.
             The dotted line is the calculation without correlation effects;
             the dashed line, with the effects of correlation between undetected nucleons;
             the dash-dotted line, with the effects of correlation between knocked-out proton and undetected nucleons;
             and solid line, with over-all correlation effects.
             The figure on the right includes CT.
             The dashed line is the calculation without correlation effects;
             the solid line, with corrleation effects.
             The rise in transparency with $Q^2$ is the characteristic
             signature of the onset of CT.
             Note that the effect of nucleon correlations on the CT model is
             a correction of a few percent.
             }
    \label{fig:frankfurt_transparency}
\end{figure}



\subsection{Cosyn}
Cosyn~\cite{Cosyn_2006}. RMSGA + CT (quantum diffusion)

Similar to eikonal approximation of equation~\ref{eqn:eikonal_approximation}.
After scattering, the outgoing wavefunction $\psi_{out}$ picks up a complex
phase $\chi(\vec{r})$.
\begin{equation}
    \psi_{out}(\vec{r}) = e^{i\chi(\vec{r})} \psi_{in}(\vec{r})
\end{equation}
Cosyn et al. introduce a Gaussian \textit{grey disc} profile function
$\Gamma(\vec{b})$ to parameterize this phase as a function of impact parameter
$\vec{b}$.
For a single rescattering,
\begin{equation}
    \psi_{out}(\vec{r}) = (1-\Gamma(\vec{b})) \psi_{in}(\vec{r})
\end{equation}
The profile function for a hadron $h$ scattering from a nucleon $N$ is
determined by
the interaction cross section $\sigma_{tot}$,
the ratio of the real and imagniary parts of the scattering amplitude $\epsilon_{hN}$,
and
slope parameter $\beta_{hN}$,
all of which are momentum-dependent.
\begin{equation}
    \Gamma_i(\vec{b}) =
        \frac{\sigma_{tot}(1-i\epsilon_{hN})}
             {4\pi\beta_{hN}^2}
        \exp{-\frac{\vec{b}^2}{2\beta_{hN}^2}}
\end{equation}
Using the same frozen approximation as the Glauber approximation, the total
phase shift for $N_{scatter}$ rescatterings is simply a product
\begin{equation}
    e^{i\chi(\vec{r})} = \prod_i^{N_{scatter}} \left(1-\Gamma_i(\vec{b_i})\right)
\end{equation}


To include CT in the model, Cosyn et al. replace
the total cross section $\sigma_{tot}$
with
an effective cross section $\sigma_{eff}$.
based on the quantum diffusion model~\cite{Farrar_1988},
which accounts for reduced interaction between the prehadron and nuclear matter
over a hadron formation length $l_h$,
\begin{equation}
    \sigma_{eff} = \sigma_{tot}
    \left\{
        \left[\frac{z}{l_h} +
               \frac{\left\langle n^{2} k_{t}^{2}\right\rangle}{t} \left(1-\left(\frac{z}{l_h}\right)\right)
        \right]
        \theta\left(l_h-z\right) +
        \theta\left(z-l_h\right)
    \right\}
\end{equation}
In this expression,
$n$ is the number of valence quarks (2 for mesons, 3 for baryons),
$k_t=\SI{0.35}{\giga\electronvolt}\sim \SI{1}{\giga\electronvolt\squared}/Q^2$
is the average transverse momentum of a quark inside a hadron,
$z$ is the distance the object has traveled since its creation,
and
$l_h=2p/\Delta M^2$ is the hadronic formation length.
This length depends on
the momentum $p$ of the outgoing hadron
and
the mass squared difference between the prehadron and outgoing hadron state.
Cosyn et al. use $\Delta M^2 = \SI{1}{\giga\electronvolt\squared}$ for protons.
