\section{Nuclear Transparency}
In order to quantify the effects of final state interactions in interactions
such as quasielastic scattering, experiments measure nuclear transparency--the
ratio of
the measured interaction cross section
to
the cross section calculated in the plane wave impulse approximation (PWIA).
The definition of transparency used in this work is the same as that used by
previous experiments experiments looking for the onset of CT in
$^{12}C(e,e'p)$,
\begin{equation} \label{eqn:transparency_definition}
    T(Q^2) = \frac{\int_{V} d^{3} p_{m} d E_{m} Y^{exp }(E_{m}, \vec{p}_{m})}
                  {\int_{V} d^{3} p_{m} d E_{m} Y^{PWIA}(E_{m}, \vec{p}_{m})}
\end{equation}
where $Y^{exp}$ and $Y^{PWIA}$ are charge-normalized yields from experiment and
simulation, integrated over a volume of missing energy and momentum phase space
$V$, defined by the cuts $E_m < \SI{80}{\mega\electronvolt}$ and
$|\vec{p}_m| < \SI{300}{\mega\electronvolt}$.
Nuclear transparency can be thought of as the probability that a struck proton
will exit the nucleus without rescattering from another nucleon.
