\section{Nuclear Transparency}
In order to quantify the effects of final state interactions in interactions
such as quasielastic scattering, experiments measure nuclear transparency--the
ratio of
the measured interaction cross section
to
the cross section calculated in the PWIA.
The definition of transparency used in this work is the same as that used by
previous experiments experiments looking for the onset of CT in
${}^{12}C(e,e'p)$,
\begin{equation} \label{eqn:transparency_definition}
    T(Q^2) = \frac{\int_{V} d^{3} p_{m} d E_{m} Y^{exp }(E_{m}, \vec{p}_{m})}
                  {\int_{V} d^{3} p_{m} d E_{m} Y^{PWIA}(E_{m}, \vec{p}_{m})}
\end{equation}
where $Y^{exp}$ and $Y^{PWIA}$ are charge-normalized yields from experiment and
simulation.

The experimental yield is given by $Y=N/Q$, where $N$ is the total number of
scattering events measured per integrated beam charge $Q$ incident on the
target.
If the target density is $\rho$, its length $l$, and the beam current is $j$,
the total number of events is $N=\int_{t_1}^{t_2} j\rhoL\sigma dt$ integrated
over the total time the beam was on $t_2-t_1$.

The yields are integrated over a volume of missing energy and momentum phase
space $V$, defined by the cuts $E_m < \SI{80}{\mega\electronvolt}$ and
$|\vec{p}_m| < \SI{300}{\mega\electronvolt}$.
These cuts prevent inelastic contributions from pion production and ensure that
the recoil nucleus remains in its ground state.
Nuclear transparency can be thought of as the probability that a struck proton
will exit the nucleus without rescattering from another nucleon.
