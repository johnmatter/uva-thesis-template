\section{Simulation}
The simulated yields are taken from Monte Carlo simulations of scattering
processes, radiative effects, and spectrometer performance.
These simulations are carried out by the FORTRAN program
\textit{SIMC}~\cite{simc_github, simc_wiki}.


\textit{SIMC} was initially written for the NE18 experiment at
SLAC~\cite{Makins_1994} and subsequently adapted for the HMS and SOS
spectrometers in JLab's Hall C.
Further development added support for more scattering processes, the pair of
HRS spectrometers in Hall A, and more recently the new SHMS spectrometer in
Hall C.


\textit{SIMC} generates events over a wide phase space, starting from beam and
target geometry; propagates the events through the spectrometers, accounting
for final state interactions, energy loss, multiple scattering, and
spectrometer acceptance and resolution; reconstructs target variables from
tracks fit at the focal plane; then applies a weight based on a model cross
section for the initial kinematics of each event.


%------------------
% TODO: move most of this to chap 4 or an appendix?

The general structure of SIMC's simulation for (quasi)elastic
scattering is described below.
\begin{enumerate}
    \item Initialization
    \begin{itemize}
        \item Choose a reaction and final state
        % \item Disable/enable implementation of (or correction for) raster, eloss ...
    \end{itemize}

    \item Event Generation
    \begin{itemize}
        \item Generate an event vertex based on target geometry,
              beam width, beam raster, and beam energy
        \item Generate $\theta_e$, $\phi_e$, $p_e$,
                       $\theta_p$, $\phi_p$, $p_p$
    \end{itemize}

    \item Event Propagation
    \begin{itemize}
        \item Adjust event for radiative effects, Coulomb corrections, particle decays, etc.
        \item Propagate particles through spectrometers using COSY models,
              applying energy loss and multiple scattering in the target and
              detectors.
    \end{itemize}

    \item Event Reconstruction
    \begin{itemize}
        \item Fit tracks in the focal plane
        \item Reconstruct target variables $\delta$, $x'_{tar}$, $y'_{tar}$, $y_{tar}$
    \end{itemize}

    \item Normalization and Saving to Disk
    \begin{itemize}
        \item Calculate $E_m$ and $\vec{p}_m$ if simulating quasielastic scattering
        \item Calculate per-event weight from spectral function, cross section,
              radiative correction weight, and event generation weight.
        \item Calculate normalization factor \textit{normfac} from luminosity,
              phase space factors, and total number of events generated.
        \item Save per-event physics quantities and weights in an Ntuple in a
              PAW HBOOK.
    \end{itemize}
\end{enumerate}

\begin{equation}
    \text{normfac} = \frac{\mathcal{L} \Delta E_p \Delta \Omega_p \Delta E_e \Delta \Omega_e}
                          {N_{gen}}
\end{equation}

\textit{SIMC}'s output \texttt{.hbook} files can be converted to \texttt{.root}
files for analysis alongside \textit{hcana} output using the utility
\textit{h2root}.

% Good note explaining normfac
% https://hallaweb.jlab.org/12GeV/experiment/E12-07-108/Publications/Technical/Spectrometer/SIMC/simc_extra.pdf
% Gaskell/Arrington talk on simc
% https://hallaweb.jlab.org/collab/meeting/2009-winter/talks/Analysis%20Workshop%20--%20Dec%2014/simc_overview.pdf

The spectral functions used in SIMC are based on the independent particle shell
model (IPSM), which assumes nucleons occupy shells with quantum numbers $n$,
$l$, $j$, similar to the model of electron orbitals in atomic physics.
% preetty good slides
% http://indico.ictp.it/event/7641/session/21/contribution/46/material/0/0.pdf
In this model, the spectral function can be factored into a sum of per-shell
energy and momentum distributions
\begin{equation}
    S(E_m,\vec{p}_m) = \sum_i N_i \norm{\varphi_i(\vec{p})}^2 L_i(E_m)
\end{equation}
where $N_i$ is the occupation number of the $i$th shell,
$\varphi_i(\vec{p})$ is the bound state wavefunction,
and $L_i(E_m)$ is an energy profile.


% TODO: Give one of these Ls a tilde or prime to account for normalization?
The energy profile of each nuclear shell $i$ with binding energy $E_i$ is
given by a Lorentzian with finite width $\Gamma_i$ that accounts for the finite
lifetime of the one-hole state.
\begin{equation}
    L_i(E) = \frac{1}{\pi} \frac{\Gamma/2}{(E-E_i)^2 + \Gamma^2/4}
\end{equation}

The separation energy $E$ cannot be less than the minimum proton removal
energy $E_{min}=m_p + m_{A-1} - m_A$, so these profiles are cut off below
$E_min$ and normalized to ensure the spectroscopic sum rule,
Equation~\ref{eqn:spectroscopic_sum_rule}, is obeyed.
\begin{equation}
    L_i(E) =
    \begin{cases}
        L_i(E) / \int^{\infty}_{E_{min}} dE L_i(E) & \text{if $E \geq E_{min}$} \\
        0 & \text{if $E<E_{min}$}
    \end{cases}
\end{equation}


The wavefunctions $\varphi_i(\vec{p})$ are solutions to the Schroedinger
equation with a sum of
a Coulomb potential $V_C$,
Woods-Saxon potential $-V_0 f(\mathbf{\vec{r}}$,
and spin-orbit coupling,
\begin{equation}
    V(\vec{r}) = -V_{0} f(r)
                 + V_{SO} \left(\frac{\hbar}{m_\pi c}\right)^{2}
                          \frac{2}{r} \frac{df}{dr}
                          \vec{l}\cdot\vec{s}
                 + V_{C}(r)
\end{equation}




% TODO: describe FSI etc in some detail
