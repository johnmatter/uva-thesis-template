\section{This Experiment: E12-06-107}
Previous measurements of nuclear transparency in quasielastic electron
scattering experiments have been consistent with the Glauber prediction, as
shown in Fig~\ref{fig:c12eep_transparency_intro}.
The goal of this experiment was to extend the range of $Q^2$ studied
in quasielastic ${}^{12}C(e,e'p)$ scattering in hopes of observing the onset
of CT.
Data were taken in Hall C at the Thomas Jefferson National Accelerator Facility
in Newport News, VA, using the High Momentum Spectrometer (HMS) and new Super
High Momentum Spectrometer (SHMS) in coincidence.
Data were taken with carbon foil and liquid hydrogen targets over a range of
momentum transfer $Q^2$ from 8.0 to \SI{14.2}{\giga\electronvolt\squared}.
The spectrometer angles and central momenta for these $Q^2$ points are listed in Table~\ref{tab:E1206107_kinematics}.

% TODO: Do we have a PDF version of this?
\begin{figure}[!h]
    \centering
    \includegraphics[width=0.6\textwidth]{chap1/c12eep_measurements_and_predictions.png}
    \caption[Transparency measurements from several experiments studying
             quasielastic electron scattering carbon.]{Transparency measurements from several experiments studying
             quasielastic electron scattering carbon.
             Data taken at JLab~\cite{Abbot_1998, Garrow_2002, Rohe_2005} are shown as squares.
             Data taken at SLAC~\cite{Makins_1994, ONeill_1995} are shown as solid triangles.
             Data taken at Bates~\cite{Garino_1992} are shown as open circles.
             The $Q^2$ locations of this experiment's measurements are shown as
             red circles with arbitrary $T$ values and error bars represented
             expected uncertainty.
             The solid red line is a Glauber calculation from~\cite{Pandharipande_1992} for carbon data.
             The solid blue line is the prediction of Cosyn et al.'s
             relativistic Glauber model~\cite{Cosyn_2006, Cosyn_2008}.
             The dashed blue lines are the predictions of Frankfurt et al.'s
             Glauber model~\cite{Frankfurt_1995_PRC} that includes the effects of CT for three choices of
             parameters.
            }
    \label{fig:c12eep_transparency_intro}
\end{figure}

\begin{table}[h]
    \centering
    \caption{
            The kinematic settings used in the E12-06-107 experiment in Hall C at JLab.
            }
    \begin{tabular}{ccccc}
\specialrule{.1em}{.05em}{.05em}
            % $Q^2$ (\si{\giga\electronvolt\squared}) & SHMS angle (${}^\circ$) & SHMS central $p$ (\si{\giga\electronvolt}) & HMS angle (${}^\circ$°) & HMS central $p$ (\si{\giga\electronvolt}) \\
            $Q^2$ (\si{\giga\electronvolt\squared}) & $\theta_{SHMS}$ (${}^\circ$) & $p_{SHMS}$ (\si{\giga\electronvolt}) & $\theta_{HMS}$ (${}^\circ$°) & $p_{HMS}$ (\si{\giga\electronvolt}) \\
\specialrule{.1em}{.05em}{.05em}
            8.0                                     & 17.1                         & 5.122                                & 45.1                         & 2.131                                     \\
            9.5                                     & 21.6                         & 5.925                                & 23.2                         & 5.539                                     \\
            11.5                                    & 17.8                         & 7.001                                & 28.5                         & 4.478                                     \\
            14.2                                    & 12.8                         & 8.505                                & 39.3                         & 2.982                                     \\
\specialrule{.1em}{.05em}{.05em}
    \end{tabular}
    \label{tab:E1206107_kinematics}
\end{table}

Chapter 2 contains an overview of theoretical considerations relevant to
the experiment and a brief history of previous experiments that have studied
color transparency.
Chapters 3 and 4 describe the experimental apparatus and data analysis
procedure.
Chapter 5 contains the final measurements of nuclear transparency and missing
energy and momentum.
Chapter 6 is a conclusion and summary.
