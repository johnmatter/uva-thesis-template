\section{Efficiency}
In general, the efficiency of a given detector system is estimated by using
information from other systems to select a set of ``clean'' events that can
reasonably be expected to produce a signal in that particular system.
Let $N_{should}$ be the number of events selected by a cut $C_{should}$ that
should show a signal in detector $D$, and $N_{did}$ be the subset of those
events that pass a cut $C_{did} = C_{should} \wedge C_{D}$,
where $C_{D}$ is an additional cut on information from the detector
$D$ under consideration and $\wedge$ represents the logical operation
\textit{and}.
Then the efficiency of the $D$ is $\epsilon_D = N_{did}/N_{should}$.

\subsection{Calorimeters and Cherenkovs}

\subsection{Tracking}

\subsection{Trigger}

\subsection{Livetime}
When a trigger is accepted, the DAQ system is unable to accept additional
triggers for a time determined by the gate widths of front end electronics and
the time it takes for CODA to write the event to disk.
The time during which the DAQ is unable to accept another event is referred to
as deadtime.
The inverse concept, livetime, refers to the total time that the DAQ is
\textit{not} occupied by processing incoming triggers.
The total livetime $TLT$ is a correction applied to the charge normalized yield
to account for events that are missed because of this phenomenon.

% TODO: edtm logic diagram
The EDTM (see Section~\ref{sec:edtm}) pulser allows an estimate of the total
livetime in a given run.
It sends regular pulses at a low frequency (\SI{3}{\hertz} in our experiment) at the
trigger logic level in the counting house.
By comparing the number of triggers that are accepted by the DAQ to the number
of pulses that are counted by the EDTM scaler, we can estimate the total
livetime as
\begin{equation}
    TLT = \frac{N_{EDTM,accepted}}{N_{EDTM,scaler}}
\end{equation}
where $N_{EDTM,accepted}$ is the number of events with a non-zero hit in the
EDTM TDC spectrum and $N_{EDTM,scaler}$ is the total number of EDTM triggers
counted by the EDTM scaler.
% TODO: Peter Bosted's correction. I don't understand his derivation.
% Let n be accepted, N be scaler count. Then L=n/N
% Let j be the current, and jc be the "cut current"
% Let f=j/jc
% Then Peter's correction is Lnew = (L-(1-f))/f
% The assumption here is that livetime is 100% when the beam is "off"
% i.e. below the cut value (typically a couple uA) used by hcana


% TODO: trigger rate figure
Because livetime should be dependent on trigger rate and the trigger rates
vary between our kinematic settings, we calculate a per-run livetime to
correct each run's yield.
Moreover, the EDTM system was not functional when we took our first set of data
for $Q^2$ of \SI{8}{\giga\electronvolt\squared}.
To estimate the livetime for these data, we performed a linear fit of the other
kinematics' $TLT$ dependence on SHMS 3/4 trigger rate and used this fit to
estimate the livetime for the $Q^2$=\SI{8}{\giga\electronvolt\squared} runs.

% TODO: LTE vs pTRIG1 (SHMS 3/4)

The computer livetime can be estimated as
\begin{equation}
    TLT = \frac{N_{phys,accepted}-N_{EDTM,accepted}}{N_{phys,scaler}-N_{EDTM,scaler}}
\end{equation}
The computer livetime for our data is neglible because CODA was configured to
only take coincidence events, whose rates were all quite low (below
\SI{6}{\hertz}) for all our kinematics.

% TODO: computer livetime vs pTRIG6

% TODO: add refs for Eric and Dave's slides

