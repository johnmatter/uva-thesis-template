\section{Efficiency}
In general, the efficiency of a given detector system is estimated by using
information from other systems to select a set of ``clean'' events that can
reasonably be expected to produce a signal in that particular system.
Let $N^{should}$ be the number of events selected by a cut $C^{should}$ that
should show a signal in detector $D$, and $N^{did}$ be the subset of those
events that pass a cut $C^{did} = C^{should} \land C^{D}$,
where $C^{D}$ is an additional cut on information from the detector
$D$ under consideration and $\land$ represents the logical operation
\textit{and}.
Then the efficiency of $D$ is $\epsilon^D = N^{did}/N^{should}$.

\subsection{HMS Calorimeter} \label{sec:hcal_eff}
A set of electrons that should have normalized track energy deposition
approximately equal to one are selected using the cuts in
Table~\ref{tab:hcal_cuts}.

\begin{table}[h]
    \centering
    \caption{List of cuts used to estimate HMS Calorimeter efficiency.}
    \label{tab:hcal_cuts}
    \begin{tabular}[t]{| c | l | l |}
        \hline
                   &  Variables              &  Cut \\ \hline
        \hline
        \multirow{4}{*}{\makecell[ml]{$C^{should}$}}
        &  HMS Cherenkov NPE      &  H.cer.npeSum>0                         \\ \cline{2-3}
        &  $\delta_{HMS}$         &  -10.0 < H.gtr.dp \&\& H.gtr.dp < 10.0  \\ \cline{2-3}
        &  $\beta_{HMS}$          &  0.8 < H.gtr.beta \&\& H.gtr.beta < 1.2 \\ \cline{2-3}
        &  Good hodoscope time    &  H.hod.goodstarttime==1                 \\ \hline
        \hline
        \multirow{2}{*}{\makecell[ml]{$C^{HCal}$}}
        &  HMS Calorimeter Energy &  \makecell{0.8 < H.cal.etottracknorm \&\&  \\
                                               H.cal.etottracknorm < 1.15} \\ \hline
    \end{tabular}
\end{table}

Events are divided into $\delta$ bins of width 5 and efficiencies
$\epsilon_i=N_i^{did}/N_i^{should}$ are calculated for each bin.
A 95\% confidence interval for each bin's estimated efficiency is obtained
using the Clopper-Pearson method.
% TODO: This is a little unclear; sigma should be the difference between epsilon and the bounds
A weight $w_i=1/\sigma_i^2$ for each bin is assigned, where $\sigma_i$ is the
larger of differences between $\epsilon_i$ and the upper and lower CI bounds.
Then the weighted efficiency is
\begin{equation}
    \epsilon = \frac{\sum_i w_i \epsilon_i}
                    {\sum_j w_j}
\end{equation}
with uncertainty
\begin{equation}
    \sigma_\epsilon = \frac{1}{\sqrt{\sum_i w_i}}
\end{equation}

\subsection{HMS Cherenkov}
A set of electrons that should fire the Cherenkov
are selected using the cuts in
Table~\ref{tab:hcer_cuts}.
A weighted efficiency is calculated as discussed in Section~\ref{sec:hcal_eff}

\begin{table}[h]
    \centering
    \caption{List of cuts used to estimate HMS Chereknov efficiency.}
    \label{tab:hcer_cuts}
    \begin{tabular}[t]{| c | l | l |}
        \hline
                   &  Variables              &  Cut \\ \hline
        \hline
        \multirow{5}{*}{\makecell[ml]{$C^{should}$}}
        &  HMS Calorimeter Energy &  \makecell{0.8 < H.cal.etottracknorm \&\&  \\
                                               H.cal.etottracknorm < 1.15} \\ \cline{2-3}
        &  $\delta_{HMS}$         &  -10.0 < H.gtr.dp \&\& H.gtr.dp < 10.0  \\ \cline{2-3}
        &  $\beta_{HMS}$          &  0.8 < H.gtr.beta \&\& H.gtr.beta < 1.2 \\ \cline{2-3}
        &  Good hodoscope time    &  H.hod.goodstarttime==1                 \\ \hline
        \hline
        \multirow{1}{*}{\makecell[ml]{$C^{HCer}$}}
        &  HMS Cherenkov NPE      &  H.cer.npeSum>0                         \\ \hline
    \end{tabular}
\end{table}

% TODO: mirror
Discuss how we dealt with the broken mirror.

\subsection{SHMS Noble Gas Cherenkov}
The SHMS Noble Gas Cherenkov was used as a pion veto.
The SHMS central momenta used in this experiment range from
5.122 to \SI{8.505}{\giga\electronvolt}.
The pion threshold in this detector is \SI{4.65}{\giga\electronvolt} while the
the proton threshold is \SI{999999999}{\giga\electronvolt}.
A set of protons that should not fire the Cherenkov
are selected using the cuts in Table~\ref{tab:pcer_cuts}.
A weighted efficiency is calculated as discussed in Section~\ref{sec:hcal_eff}

\begin{table}[h]
    \centering
    \caption{List of cuts used to estimate SHMS Noble Gas Chereknov efficiency.}
    \label{tab:pcer_cuts}
    \begin{tabular}[t]{| c | l | l |}
        \hline
                   &  Variables              &  Cut \\ \hline
        \hline
        \multirow{3}{*}{\makecell[ml]{$C^{should}$}}
        &  $\delta_{SHMS}$        &  -10.0 < P.gtr.dp \&\& P.gtr.dp < 10.0  \\ \cline{2-3}
        &  $\beta_{SHMS}$         &  P.gtr.beta < 1.4 \\ \cline{2-3}
        &  Good hodoscope time    &  P.hod.goodstarttime==1                 \\ \hline
        \hline
        \multirow{1}{*}{\makecell[ml]{$C^{PCer}$}}
        &  SHMS Cherenkov NPE     &  P.ngcer.npeSum<0.1                     \\ \hline
    \end{tabular}
\end{table}

\subsection{Tracking}

\begin{table}[h]
    \centering
    \caption{List of cuts used to estimate HMS tracking efficiency.}
    \label{tab:htrack_cuts}
    \begin{tabular}[t]{| c | l | l |}
        \hline
                   &  Variables              &  Cut \\ \hline
        \hline
        \multirow{11}{*}{\makecell[ml]{$C^{should}$}}
        & Fiducial cut              & H.hod.goodscinhit==1 \\ \cline{2-3}
        & $\beta_{notrack}$         & \makecell{0.5 < H.hod.betanotrack \&\& \\
                                                H.hod.betanotrack < 1.4} \\ \cline{2-3}
        & Few hits in DC 1          & \makecell{(H.dc.1x1.nhit + H.dc.1u2.nhit + \\
                                                 H.dc.1u1.nhit + H.dc.1v1.nhit + \\
                                                 H.dc.1x2.nhit + H.dc.1v2.nhit) < 35} \\ \cline{2-3}
        & Few hits in DC 1          & \makecell{(H.dc.2x1.nhit + H.dc.2u2.nhit + \\
                                                 H.dc.2u1.nhit + H.dc.2v1.nhit + \\
                                                 H.dc.2x2.nhit + H.dc.2v2.nhit) < 35} \\ \cline{2-3}
        & SHMS Cherenkov NPE        & P.ngcer.npeSum<0.1 \\ \cline{2-3}
        & HMS Cherenkov NPE         & H.cer.npeSum>0 \\ \hline

        \multirow{1}{*}{\makecell[ml]{$C^{HTrack}$}}
        & At least one track found  & H.dc.ntrack>0 \\ \hline
    \end{tabular}
\end{table}

\begin{table}[h]
    \centering
    \caption{List of cuts used to estimate SHMS tracking efficiency.}
    \label{tab:htrack_cuts}
    \begin{tabular}[t]{| c | l | l |}
        \hline
                   &  Variables              &  Cut \\ \hline
        \hline
        \multirow{11}{*}{\makecell[ml]{$C^{should}$}}
        & Fiducial cut              & P.hod.goodscinhit==1 \\ \cline{2-3}
        & Good hodoscope time       & P.hod.goodstarttime==1 \\ \cline{2-3}
        & $\beta_{notrack}$         & P.hod.betanotrack < 1.2 \\ \cline{2-3}
        & Few hits in DC 1          & \makecell{(P.dc.1x1.nhit + P.dc.1u2.nhit + \\
                                                 P.dc.1u1.nhit + P.dc.1v1.nhit + \\
                                                 P.dc.1x2.nhit + P.dc.1v2.nhit) < 25} \\ \cline{2-3}
        & Few hits in DC 1          & \makecell{(P.dc.2x1.nhit + P.dc.2u2.nhit + \\
                                                 P.dc.2u1.nhit + P.dc.2v1.nhit + \\
                                                 P.dc.2x2.nhit + P.dc.2v2.nhit) < 25} \\ \cline{2-3}
        & SHMS Cherenkov NPE        & P.ngcer.npeSum<0.1 \\ \cline{2-3}
        & HMS Cherenkov NPE         & H.cer.npeSum>0 \\ \hline

        \multirow{1}{*}{\makecell[ml]{$C^{PSingleTrack}$}}
        & One track found           & P.dc.ntrack==1 \\ \hline


        \multirow{17}{*}{\makecell[ml]{$C^{PMultipleTrack}$}}
        & \makecell{More than one track \\ found} & P.dc.ntrack > 1 \\ \cline{2-3}
        & \makecell{Few hits on negative \\ side ADCs} & 
                \makecell{P.hod.1x.totNumGoodNegAdcHits<5 \&\& \\ 
                          P.hod.1y.totNumGoodNegAdcHits<5 \&\& \\
                          P.hod.2x.totNumGoodNegAdcHits<5 \&\& \\
                          P.hod.2y.totNumGoodNegAdcHits<5} \\ \cline{2-3}
        & Good focal plane time     &
                \makecell{-10 < P.hod.1x.fptime      \&\& \\
                                P.hod.1x.fptime < 50 \&\& \\
                          -10 < P.hod.1y.fptime      \&\& \\
                                P.hod.1y.fptime < 50 \&\& \\
                          -10 < P.hod.2x.fptime      \&\& \\
                                P.hod.2x.fptime < 50 \&\& \\
                          -10 < P.hod.2y.fptime      \&\& \\
                                P.hod.2y.fptime < 50} \\ \cline{2-3}
        & $\delta$                   & -15 < P.gtr.dp \&\& P.gtr.dp < 15 \\ \cline{2-3}
        & $y_{tar}$                  & -5 < P.gtr.y \&\& P.gtr.y < 5 \\ \cline{2-3}
        & $x'_{tar}$                & -0.2 < P.gtr.th \&\& P.gtr.th < 0.2 \\ \cline{2-3}
        & $y'_{tar}$                & -0.2 < P.gtr.ph \&\& P.gtr.ph < 0.2 \\ \hline
    \end{tabular}
\end{table}

\subsection{Trigger}

\subsection{Livetime} \label{sec:livetime}
When a trigger is accepted, the DAQ system is unable to accept additional
triggers for a time determined by the gate widths of front end electronics and
the time it takes for CODA to write the event to disk.
The time during which the DAQ is unable to accept another event is referred to
as deadtime.
The inverse concept, livetime, refers to the total time that the DAQ is
\textit{not} occupied by processing incoming triggers.
The total livetime $TLT$ is a correction applied to the charge normalized yield
to account for events that are missed because of this phenomenon.

% TODO: edtm logic diagram
The EDTM (see Section~\ref{sec:edtm}) pulser allows an estimate of the total
livetime in a given run.
It sends regular pulses at a low frequency (\SI{3}{\hertz} in our experiment) at the
trigger logic level in the counting house.
By comparing the number of triggers that are accepted by the DAQ to the number
of pulses that are counted by the EDTM scaler, we can estimate the total
livetime as
\begin{equation}
    TLT = \frac{N_{EDTM,accepted}}{N_{EDTM,scaler}}
\end{equation}
where $N_{EDTM,accepted}$ is the number of events with a non-zero hit in the
EDTM TDC spectrum and $N_{EDTM,scaler}$ is the total number of EDTM triggers
counted by the EDTM scaler.
% TODO: Peter Bosted's correction. I don't understand his derivation.
% Let n be accepted, N be scaler count. Then L=n/N
% Let j be the current, and jc be the "cut current"
% Let f=j/jc
% Then Peter's correction is Lnew = (L-(1-f))/f
% The assumption here is that livetime is 100% when the beam is "off"
% i.e. below the cut value (typically a couple uA) used by hcana


% TODO: trigger rate figure
Because livetime should be dependent on trigger rate and the trigger rates
vary between our kinematic settings, we calculate a per-run livetime to
correct each run's yield.
Moreover, the EDTM system was not functional when we took our first set of data
for $Q^2$ of \SI{8}{\giga\electronvolt\squared}.
To estimate the livetime for these data, we performed a linear fit of the other
kinematics' $TLT$ dependence on SHMS 3/4 trigger rate and used this fit to
estimate the livetime for the $Q^2$=\SI{8}{\giga\electronvolt\squared} runs.

% TODO: LTE vs pTRIG1 (SHMS 3/4)

The computer livetime can be estimated as
\begin{equation}
    TLT = \frac{N_{phys,accepted}-N_{EDTM,accepted}}{N_{phys,scaler}-N_{EDTM,scaler}}
\end{equation}
The computer livetime for our data is negligible because CODA was configured to
only take coincidence events, whose rates were all quite low (below
\SI{6}{\hertz}) for all our kinematics.

% TODO: computer livetime vs pTRIG6

% TODO: add refs for Eric and Dave's slides

