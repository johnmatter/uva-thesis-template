\section{Color Transparency}
Color transparency (CT) refers to vanishing initial and final state
interactions in exclusive processe at large momentum transfer $Q^2$
between hadrons and the nuclear medium.
Mueller~\cite{Mueller_1981} and Brodsky~\cite{Brodsky_1982} first proposed the
existence of CT in the early 1980s.
At large $Q^2$, the quarks inside a hadron can form an object of small
transverse size that does not radiate gluons.
If this small transverse size is maintained for distances comparable to the
size of the nucleus, the hadron passes through the medium without further
interactions.
These criteria can be summarized as squeezing (small transverse size) and
freezing (maintaining that size over distances comparable to the size of the
nucleus).

QED dipole analogy.

Relation to QCD factorization theorems and Bjorken scaling

A more in-depth discussion of these considerations can be found in recent
reviews by Dutta, Hafidi, and Strikman~\cite{Dutta_2013,Dutta_2012}.
