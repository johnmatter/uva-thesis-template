\chapter{Summary and Conclusion}
Using the upgraded \SI{12}{\giga\electronvolt} CEBAF beam at JLab, coincidence
$(e,e'p)$ data were taken with $^{1}H$ and $^{12}C$ targets for $Q^2$ values
between 8.0 and \SI{14.2}{\giga\electronvolt\squared}.
Nuclear transparencies were extracted for each kinematic point by
taking the ratio of charge-normalized yields from experiment to yields from
Monte Carlo PWIA simulation.
The transparency measured at the lowest kinematic point at
$Q^2=\SI{8.0}{\giga\electronvolt\squared}$ agrees with prior measurements at
JLab.
The $Q^2$-dependence of the measured transparencies is consistent with
traditional Glauber multiple scattering theory and does not show an onset of
color transparency in $^{12}C(e,e'p)$ up to
$Q^2=\SI{14.2}{\giga\electronvolt\squared}$.
% Though other work has suggested that the onset may be more obvious in the
% $s$-shell than in the $p$-shell, this effect is not evident in this
% experiment's results.
Given the onset of CT's connection to factorization theorems, this result may
be troubling for efforts to study baryon GPDs.


As discussed in Sec~\ref{sec:ct_def}, Brodsky and de Téramond~\cite{Brodsky_2021}
use light-front holographic QCD to derive an expression for a PLCs transverse
size as a function of twist $\tau$ and $Q^2$.
Their calculations suggest that the onset of CT in ${}^{12}C(e,e'p)$ may be
higher than what can currently be probed in Hall C, perhaps not occurring
until $Q^2\sim\SI{20}{\giga\electronvolt\squared}$.


Using the same framework, Caplow-Munro and Miller~\cite{CaplowMunro_2021}
demonstrate that expansion effects are not sufficiently large enough to credit
final state interactions with the absence of CT in this experiment's
measurements.
If a PLC were formed, it should have remained small as it exited the nucleus.
The lack of a rise in transparency then suggests that a PLC was not formed,
and that the proton's electromagnetic form factor is dominated at large
momentum by the Feynman mechanism\footnote{In the Feynman mechanism, a single
quark carrying a large portion of a hadron's momentum absorbs the entire
momentum transfer $Q$~\cite{Drell_1970}.} rather than a PLC.


Experiments at JLab in the near future will include color transparency studies
in $\rho$ (Hall B) and $\pi$ (Hall C) electroproduction, and $\pi$
photoproduction in Hall D.
A future $\sim\SI{20}{\giga\electronvolt}$ upgrade to the CEBAF accelerator is
possible~\cite{Bogacz_2020}.
Replacing the highest-energy recirculating arcs with Fixed Field Alternating
Gradient arcs would allow 6--7 additional beam passes through the existing
CEBAF SRF cavities, yielding a beam with nearly double the present maximum
energy.
With this upgrade, a future experiment could revisit the results of this
experiment and probe a range of $Q^2$ that Brodsky et al. suggest may finally
show the onset of color transparency.
