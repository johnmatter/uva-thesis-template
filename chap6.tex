\chapter{Summary and Conclusion}
Using the upgraded 12\,\,GeV CEBAF beam at JLab, coincidence $(e,e'p)$ data
were taken with $^{1}H$ and $^{12}C$ targets for $Q^2$ values between 8 and
14.2\,(GeV/$c)^2$.
Nuclear transparencies were extracted for each kinematic point by integrating
charge-normalized yields and taking their ratio.
The transparency measured at the lowest kinematic point at
$Q^2=8.1$\,(GeV/$c)^2$ agrees with prior measurements at JLab.
The $Q^2$-dependence of the measured transparencies is consistent with
traditional Glauber multiple scattering theory and does not show an onset of
color transparency in $^{12}C(e,e'p)$ below $Q^2=14.2$\,(GeV/$c)^2$.
% Though other work has suggested that the onset may be more obvious in the
% $s$-shell than in the $p$-shell, this effect is not evident in this
% experiment's results.

As discussed in Sec~\ref{sec:ct_def}, Brodsky and de Téramond~\cite{Brodsky_2021}
use light-front holographic QCD to derive an expression for a PLCs transverse
size as a function of twist $\tau$ and $Q^2$.
Their calculations suggest that the onset of CT in ${}^{12}C(e,e'p)$ may be
higher than what can currently be probed in Hall C, perhaps not occurring
until $Q^2=\SI{20}{\giga\electronvolt\squared}$.

Using the same framework, Caplow-Munro and Miller~\cite{CaplowMunro_2021}
demonstrate that expansion effects are not sufficiently large to cause FSI in
this experiment's measurements.
The lack of a rise in transparency then suggests that a PLC was not formed,
and that the proton's electromagnetic form factor is dominated at large
momentum by the Feynman mechanism\footnote{In the Feynman mechanism, a single
quark carrying a large portion of a hadron's momentum absorbs the entire
momentum transfer $Q$~\cite{Drell_1970}.} rather than a PLC.

Experiments at JLab in the near future will include color transparency studies in $\rho$ (Hall B) and $\pi$ (Hall C) electroproduction, and $\pi$ photoproduction in Hall D.
